\documentclass[twocolumn]{article}

\usepackage{amsmath,dsfont}
\usepackage[vlined,algoruled]{algorithm2e}

\author{Christopher Martin\\{\tt chris.martin@gatech.edu}}
\date{}

\title{
$\nat \leftrightarrow \nat^2$ bijection
for enumerating cardinality-$2$ subsets
}

\newcommand\nat{\ensuremath{\mathds{N}}}
\newcommand\abs[1]{\ensuremath{\left\vert{#1}\right\vert}}
\newcommand\closedOpen[2]{\ensuremath{\left[\,#1,#2\,\right)}}
\newcommand\bigO[1]{\ensuremath{\mathcal{O}\left(#1\right)}}
\newcommand\bracket[1]{\ensuremath{\left\{#1\right\}}}
\newcommand\arraysub[2]{\ensuremath{#1\!\left[\,#2\,\right]}}

\SetKw{Yield}{yield}

\begin{document}

\maketitle

\section{Motivation}

Iteration over the distinct unordered cardinality-2 subsets
of some finite set occurs commonly in algorithms
(``$\forall\;\text{distinct pairs}\;(a\neq b)\in S,\text{ do }\ldots$'').

With $S:\abs{S}=n$ represented by an array-like data
structure with constant-time $\arraysub{S}{a}$ lookups,
this has the trivial procedural implementation
shown in Algorithm \ref{alg:double-loop}.

\begin{algorithm}
  \caption{
  \label{alg:double-loop}
    Typical double-loop structure
    for iterating over pairs.
  }
  \For{
    $a\in\closedOpen{0}{n}$\;
  }{
    \For{
      $b\in\closedOpen{0}{i}$\;
    }{
      \Yield $\arraysub{S}{a}, \arraysub{S}{b}$\;
    }
  }
\end{algorithm}

We will show how to flatten this double loop into
$\psi(n)\equiv\frac{1}{2}(n)(n-1)$
iterations of a single loop with counter $i$
by defining a mapping $P : i \rightarrow \{a \times b\}$
that is computable in near-constant time.

$P$ gives us direct access to a well-defined ordering of
pairs, and the modified loop structure given as
Algorithm \ref{alg:single-loop}
and is more amenable to parallelization.

\begin{algorithm}
  \caption{Proposed single-loop structure.}
  \label{alg:single-loop}
  \For{
    $i\in\closedOpen{0}{\psi(n)}$\;
  }{
    $(a,b) \gets P(i)$\;
    \Yield $\arraysub{S}{a}, \arraysub{S}{b}$\;
  }
\end{algorithm}

\end{document}
